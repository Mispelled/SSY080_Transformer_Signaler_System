\section*{3.2 Linjära System och Sinusar}
\subsection*{a)}
Givet i labpm vinns ekvation 12 enligt: 
$G(s) = \frac{(s+0.1)(s+10)}{(s+1)(s^2+s+9)} =
\frac{s^2+10.1s+1}{s^3+2s^2+10s+9}$
detta polynom kan tecknas som täljare och nämnare i Matlab - samt ovandlas till
en överföringsfunktion - med följande kod:

\begin{lstlisting}
num = [1 10.1 1];
den = [1 2 10 9];
Gs=tf(num, den);
\end{lstlisting}

Bode-diagramet samt systemets pol- och nollställen kan ses i
figur~\ref{fig:task2a-bode.png} och figur~\ref{fig:task2a-pzmap.png}.

\subsection*{b)}
De tre sinussignalerna ($x1 = sin(t), x2 = sin(3t), x3 = sin(5t)$) finns att
beskåda i figur~\ref{fig:task2b}

\subsection*{c)}
Efter att ha låtit signalerna passera genom vårt givna system ($G(s)$) erhöll
vi tre nya signaler ($y1, y2, y3$) vilka hade förändrad amplitud och fas
gentemot får insignal. Detta var givetvis väntat. Den svarta kurvan visar
insignalen och den blå visar signalen vilken returnerades av lsim. Vidare
beräknade vi även amplitud och fas var och en för sig varefter vi lät den
signalen gå genom systemet för att erhålla en signal lika med den som erhölls
av lsim.

Koden för att bekräfta ekv 2 kan ses nedan:

\begin{lstlisting}
phi1 = angle(evalfr(Gs,1j));
x1p = sin(t+phi1);
y1p = abs(evalfr(Gs,1j))*x1p;

phi2 = angle(evalfr(Gs,3j));
x2p = sin(3*t+phi2);
y2p = abs(evalfr(Gs,3j))*x2p;

phi3 = angle(evalfr(Gs,5j));
x3p = sin(5*t+phi3);
y3p = abs(evalfr(Gs,5j))*x3p;
\end{lstlisting}

Här kommer $y1p, y2p \text{ samt } y3p$ vara lika med $y1, y2 \text{ och } y3$
vilka erhölls från lsim och vi ser alltså att ekv 2 stämmer.
