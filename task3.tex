\section*{3.3 Periodiska Insignaler och DFT}
\subsection*{a)}
För att generera och rita upp en fyrkantsvåg i Matlab använde vi oss av
följande kod:

\begin{lstlisting}
N = 2.^13;
F = 100;
Ts = 1/F;
t = 0:Ts:40*pi;
x = square(t);
plot(t, x)
\end{lstlisting}

Den resulterande grafen, samt dess fourierkoefficienter (vilka ska räknas ut i
senare uppgifter) finns att finna i figur~\ref{fig:task3c-square-+-fk}.

\subsection*{b)}
De tre första - nollskiljda - fourierkoefficienterna för fyrkantsvågen är: 
1.2732, 0.4244, 0.2546 och beräknades i enlighet med ekvation 6 i labpm.

\subsection*{c)}
Genom att använda den snabba fouriertransformen (den inbyggda matlabfunktionen
\emph{fft}) kunde vi beräkna den diskreta fouriertransformen av fyrkantsvågen.
Grafen till uppgiften återfinns i figur~\ref{task3c-square-+-fk} vilket är
samma figur som i uppgift a).

\subsection*{d)}
Genom att använda sambandet i ekvation 10 i labpm kunde vi beräkna de praktiska
värdena för fourierkoefficienterna vi tog fram i uppgift b. Resultatet för de
båda ges i tabellen nedan.

\begin{tabular}{| l | l | l | l |}
    \hline
     & 1 & 3 & 5 \\ \hline
    Teoretiska & 1.2732 & 0.4244 & 0.2546 \\ \hline
    Praktiska & 1.2702 & 0.4154 & 0.2398 \\ \hline
\end{tabular}

\subsection*{e)}
Fyrkantssignalen applicerades på $G(s)$ varefter de tre första
fourierkoefficienterna bestämdes teoretiskt med hjälp av ekvation 8 i labpm.
Därefter användes återigen den snabba fouriertransformen för att beräkna
koefficienterna i praktiken. En tabell med värdena ges nedan och i
figur~\ref{task3d+fk} ges återigen spektrumet för fyrkantsvågen i
frekvensdomänen samt spektrumet för utsignalen när fyrkantsvågen applicerats på
systemet $G(s)$.

\begin{tabular}{| l | l | l | l |}
    \hline
     & 1 & 3 & 5 \\ \hline
    Teoretiska & 1.1279 & 1.4020 & 0.1666 \\ \hline
    Praktiska & 1.023 & 1.3378 & 0.1513 \\ \hline
\end{tabular}
