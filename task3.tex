\section*{3.3 Periodiska Insignaler och DFT}
\subsection*{a)}
För att generera och rita upp en fyrkantsvåg i Matlab använde vi oss av
följande kod:

\begin{lstlisting}
N = 2.^13;
F = 100;
Ts = 1/F;
t = 0:Ts:40*pi;
x = square(t);
plot(t, x)
\end{lstlisting}

Den resulterande grafen, samt dess fourierkoefficienter (vilka ska räknas ut i
senare uppgifter) finns att finna i figur~\ref{fig:task3c-square-+-fk}.
Eftersom $x(t) = -x(-t)$ ser vi att signalen är udda.

\subsection*{b)}
De tre första - nollskilda - fourierkoefficienterna för fyrkantsvågen är: 
1.2732, 0.4244, 0.2546 och beräknades med hjälp av $B_k = \frac{4}{k\pi}$ där
$k=1,3,5$.

\subsection*{c)}
Genom att använda den snabba fouriertransformen (den inbyggda matlabfunktionen
\emph{fft}) kunde vi beräkna den diskreta fouriertransformen av fyrkantsvågen.
Grafen till uppgiften återfinns i figur~\ref{fig:task3c-square-+-fk} vilket är
samma figur som i uppgift a).\\\\
Spektrumet består av $\omega_k = 1, 3, 5$ och det går även att skymta $\omega_k =
7$.

\subsection*{d)}
Genom att använda sambandet i ekvation 10 ($B=\frac{2|X[K_0]|}{N}$ i labpm kunde
vi beräkna de praktiska värdena för fourierkoefficienterna vi tog fram i uppgift
b. Resultatet för de båda ges i tabellen nedan.

\begin{tabular}{| l | l | l | l |}
    \hline
     & 1 & 3 & 5 \\ \hline
    Teoretiska & 1.2732 & 0.4244 & 0.2546 \\ \hline
    Praktiska & 1.2702 & 0.4154 & 0.2398 \\ \hline
\end{tabular}

\subsection*{e)}
Fyrkantssignalen applicerades på $G(s)$ varefter de tre första
fourierkoefficienterna bestämdes teoretiskt med hjälp av ekvation 8 i labpm.
Ekvation 8 lyder som bekant: $y(t) = A_0 g(0) + \sum_{k=1}^{\infty} B_k
g(k\omega_0)sin(k\omega_0t+\Phi(k\omega_0))$. Då $g(w) = |G(jw)|$ och $B_k$ är
fourierkoefficienten för insignalen kan vi beräkna fourierkoefficienten för
utsignalen vid ett givet $k$ med hjälp av:

\begin{lstlisting}
abs(evalfr(H,kj))*(4/k*pi)
\end{lstlisting}

Därefter användes återigen den snabba fouriertransformen för att beräkna
koefficienterna i praktiken. En tabell med värdena ges nedan och i
figur~\ref{fig:task3d+fk} ges återigen spektrumet för fyrkantsvågen i
frekvensdomänen samt spektrumet för utsignalen när fyrkantsvågen applicerats på
systemet $G(s)$.\\\\
Eftersom ett system bara kan förändra en signals amplitud och fas är spektrum
oförändrat före och efter det att systemet har applicerats på signalen.

\begin{tabular}{| l | l | l | l |}
    \hline
     & 1 & 3 & 5 \\ \hline
    Teoretiska & 1.1279 & 1.4020 & 0.1666 \\ \hline
    Praktiska & 1.023 & 1.3378 & 0.1513 \\ \hline
\end{tabular}
