\section*{3.4 Notch-filter}
\subsection*{a)}
Polynomet vilket söks i uppgiften är 
$Ts = s*(s-1j)*(s+1j)*(s-5j)*(s+5j)*(s-7j)*(s+7j)*(s-9j)*(s+9j)$ där
komplexkonjugerade nollställen används för att få ett polynom med reella
koefficienter. Om man låter nämnaren vara lika med 1 och använder Matlabs
inbyggda funktion för att generera en överföringsfunktion kan det resulterande
systemet visas med hjälp av ett Bodediagramm. Detta diagramm finns att beskåda i
figur~\ref{fig:task4a-bode}. Som väntat ligger det notchar i $\omega = 1, 5, 7,
9$. Anledningen till att det är olämpligt att implementera ett filter med fler
nollställen än poler är att vi då inte får den eftersträvansvärda dämpningen
för alla frekvenser utan istället kommer förstärka vissa frekvenser och
feedbacken kommer ge en oändligt förstärkt signal.

\subsection*{b)}
Vi lade till lika många poler som vi har nollställen för att få en acceptabel
dämpning. Bodediagrammet finns att se i figur~\ref{fig:task4b-bode}. $p=-4$ är
ett bra val eftersom vi strävar efter att släppa igenom frekvenser vid
$\omega = 3$ och dämpa de övriga, särskilt de högre, frekvenserna.

\subsection*{c)}
Genom att beräkna mag2db(abs(evalfr(sys,1000j))) -
mag2db(abs(evalfr(sys,100j))) fann vi att 12 poler erofdrades för att en
dämpning på 60 dB per dekad för vinkelfrekvenser $\omega \gg 9 
\frac{\text{rad}}{s}$. Detta återspeglas även i vår slutliga uppritning av
utsignalen då färre poler ger disskontinuiteter i signalen medan 12 poler ger
en fin sinusformad kurva. Bodediagrammet som visar detta finns att beskåda i
figur~\ref{fig:task4c-bode}.\\\\
För att filtret ska bli realiserbart krävs det att det finns minst lika många
poler som det finns nollställen. Detta för att ge den eftersträvade dämpningen och inte
råka ut för feedback som resulterar i en oändlig förstärkning. I vårt fall
erfodras 10 stycken poler då vi har 10 stycken nollställen.

\subsection*{d)}
För att förstärka vårat system skalade vi täljarpolynomet med $|H(j3)|$.
Filtrets bodediagramm finns återgivet i figur~\ref{fig:task4d-bode}.

\subsection*{e)}
I figur~\ref{fig:task4e-xsignal-sys2} och figur~\ref{fig:task4e-y-sys2}
syns respektive insignal $x(t)$ i svart och utsignal $y(t)$ i blått. Man kan
tydligt se att den stationära utsignalen liknar en sinus med frekvensen $\omega
= 3$. Som brukligt kan vi även avläsa från graferna i
figur~\ref{fig:task4e-fk-x-sys2} och figur~\ref{fig:task4e-fk-y-sys2} att
amplituden är den förväntade vilket även stämmer överrens med tabellerna från
uppgift 3.3 d respektive e.
